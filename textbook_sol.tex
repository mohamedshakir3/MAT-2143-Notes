\documentclass[openany]{report}

\usepackage[utf8]{inputenc}


\usepackage{multicol}
\usepackage{stylesheet}
\usepackage{lecture_notes_styles}

\title{MAT 2143 Suggested Exercises Solutions}
\author{Last updated:}
\begin{document}
\maketitle
\tableofcontents

\chapter{Preliminaries}
\chapter{The Integers}
\chapter{Groups}
\section{Question 1}
Find all $x \in \ints$ satisfying each of the following equations. 
\begin{multicols}{2}
    \begin{enumerate}[label=(\alph*)]
        \item $3x \equiv 2 \pmod{7}$
        \item $5x + 1 \equiv 13 \pmod{23}$
        \item $5x + 1 \equiv 13 \pmod{26}$
        \item $9x \equiv 3 \pmod{5}$
        \item $5x \equiv 1 \pmod{6}$
        \item $3x \equiv 1 \pmod{6}$
    \end{enumerate}
\end{multicols}
\textbf{Solution:}
\begin{enumerate}[label=(\alph*)]
    \item $3x \equiv 2 \pmod{7}$, we have $3 \cdot 5 \equiv 1 \pmod{7}$, so $3^{-1} \equiv 5 \pmod{7}$, then 
    \[x \equiv 2 \cdot 5 \equiv 3 \pmod{7}\]
    So 
    $$x \in 3\cdot \ints$$
    \item $5x + 1 \equiv 13 \pmod{23}$ $\implies$ $5x \equiv 12 \pmod{23}$. We have $5 \cdot 14 = 70 \equiv 1 \pmod{23}$, so $5^{-1} \equiv 14 \pmod{23}$, then
    \item \[5x \equiv 12 \implies x \equiv 12 \cdot 14 \equiv 7 \pmod{23}\]
    Therefore 
    \[x \in 7 \cdot \ints\] 
    The rest of these are easy.
\end{enumerate}

\section{Question 2}
Which of the following multiplaction tables defined on the set $G = \{a,b,c,d\}$ form a group? Support your answer in each case. 
\begin{multicols}{2}
    \begin{enumerate}[label=(\alph*)]
        \item \[\begin{tabular}{c|cccc}
            $\circ$ & a & b & c & d \\
            \hline
            a & a & c & d & a \\
            b & b & b & c & d\\
            c & c & d & a & b\\
            d & d & a & b & c 
        \end{tabular}\]
        \item \[\begin{tabular}{c|cccc}
            $\circ$ & a & b & c & d \\
            \hline
            a & a & b & c & d \\
            b & b & a & d & c\\
            c & c & d & a & b\\
            d & d & c & b & a 
        \end{tabular}\]
        \item \[\begin{tabular}{c|cccc}
            $\circ$ & a & b & c & d \\
            \hline
            a & a & b & c & d \\
            b & b & c & d & a\\
            c & c & d & a & b\\
            d & d & a & b & c 
        \end{tabular}\]
        \item \[\begin{tabular}{c|cccc}
            $\circ$ & a & b & c & d \\
            \hline
            a & a & b & c & d \\
            b & b & a & c & d\\
            c & c & b & a & d\\
            d & d & d & b & c 
        \end{tabular}\]
    \end{enumerate}
\end{multicols}

\textbf{Solution:}
\begin{enumerate}[label=(\alph*)]
    \item is \emph{not} a group since there does not exist an identity. $a$ cannot be the identity since $ab = c \neq ba = b$. $b$ cannot be the identity since $bc = c \neq cb = d$. $c$ cannot be the identity since no element combined with $c$ yields that element. $d$ cannot be the identity for the same reason as $c$.
    \item is a group. Since the corresponding row and column for $a$ matches the headers, $a$ is the identity, and each element is its own inverse. We also have that $(bc)d = (d)d = a$ and $b(cd) = b(b) = a$ so the operation is associative. Therefore $(b)$ is a group.
    \item is a group. $a$ is the identity, and $a$ appears in each column and row so each element has an inverse. We also have that $(bc)d = (d)d = c$ and $b(cd) = b(b) = c$ so the operation is associative. Therefore $(c)$ is a group.
    \item is \emph{not} a group since the corresponding row and column for $a$ matches the headers, but $d$ does not have an inverse.
\end{enumerate}
\section{Question 3}
Write out Cayley tables for groups formed by the symmetries of a rectange and for $(\ints_4, +)$. How many elements are in each group? Are the groups the same? Why or why not? \\[3ex]
\textbf{Solution:} The Cayley table for the symmetry group of a rectangle is
\[\begin{tabular}{c|cccc}
    $\circ$ & $\epsilon$ & $\rho$ & $\alpha$ & $\beta$\\
    \hline
    $\epsilon$ & $\epsilon$ & $\rho$ & $\alpha$ & $\beta$\\
    $\rho$ & $\rho$ & $\epsilon$ & $\beta$ & $\alpha$\\
    $\alpha$ & $\alpha$ & $\beta$ & $\epsilon$ & $\rho$\\
    $\beta$ & $\beta$ & $\alpha$ & $\rho$ & $\epsilon$\\
\end{tabular}\]
Where $\alpha, \beta, \epsilon, \rho$ are defined as we've seen in class. The Cayley table for $(\ints_4, +)$ is
\[\begin{tabular}{c|cccc}
    + & 0 & 1 & 2 & 3\\
    \hline
    0 & 0 & 1 & 2 & 3\\
    1 & 1 & 2 & 3 & 0\\
    2 & 2 & 3 & 0 & 1\\
    3 & 3 & 0 & 1 & 2\\
\end{tabular}\]


\end{document}