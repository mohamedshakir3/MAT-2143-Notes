\documentclass[openany]{report}
\usepackage[utf8]{inputenc}
\usepackage{lecture_notes_styles}
\usepackage{stylesheet}

\renewcommand{\chaptername}{DGD}

\title{Group Theory DGD's}
\author{Last Updated:}

\begin{document}

\maketitle

\tableofcontents

\chapter{Sets, Mapping and Bijections}
\chapter{Equivalence Relations and Equivalence Classes, Well-definedness of Operations on Equivalence Classes}
\chapter{Well-defined Operations on Equivalence Classes, Examples of Groups}
\section{Question 1}
Let $n$ be some fixed positive integer, and let $X$ be the set of all $n \times n$ diagonalizable matrices. Consider each of the following equivalence relations. Do ordinary matrix addition and multiplication induce well-defined operations on the equivalence classes? 
\begin{enumerate}[label=(\alph*)]
    \item $A \sim B$ means $A = PBP^{-1}$ for some invertible $n \times n$ matrix $P$
    \item $A \sim B$ means $\det(A) = \det(B)$
\end{enumerate}
\textbf{Solution:} 
\begin{enumerate}[label=(\alph*)]
    \item To check if an operation is well-defined, we want to show the following
    \[\begin{cases}
        A \sim A' \\
        B \sim B' \\
    \end{cases} \implies A + B \sim A' + B'\]
    and 
    \[\begin{cases}
        A \sim A' \\
        B \sim B' \\
    \end{cases} \implies AB \sim A'B'\]
    Suppose $A \sim A'$ and $B \sim B'$, then $A = PA'P^{-1}$, and $B = QB'Q^{-1}$. So
    \[A + B = PA'P^{-1} + QB'Q^{-1}\]
    $A + B \sim A' + B'$ would be $A + B = R (A' + B') R'$ for some $n \times n$ invertible matrix $R$. So this would imply that $A + B$ must be a diagonalizable matrix. But, this is not always true. Consider the counter exaxmple 
    \[A = \begin{bmatrix}
        1 & 1 \\
        0 & 2
    \end{bmatrix} \text{ and } \begin{bmatrix}
        -1 & 0 \\
        0 & -2
    \end{bmatrix}\]
    $A$ and $B$ are diagonalizable since they have $n$ distinct eigenvalues. But $A + B$ is not diagonalizable. So this operation is not well-defined.\\[2ex]
    For multiplication, we have 
    \[AB = PAP^{-1}QBQ^{-1}\]
    If $P = Q$, then we get $AB = PABQ^{-1} = PABP^{-1}$ so multiplication would work, however this is not always true. Consider the counter example
    \begin{align*}
        A &= \begin{bmatrix}
            1 & 0 \\
            0 & 2
        \end{bmatrix} &&&  A' &= \begin{bmatrix}
            2 & 0 \\
            0 & 1
        \end{bmatrix} &&&
        P &= \begin{bmatrix}
            0 & 1 \\
            1 & 0
        \end{bmatrix}\\
        B &= \begin{bmatrix}
            1 & 0 \\
            0 & 2
        \end{bmatrix} &&& B' &= \begin{bmatrix}
            2 & 0 \\
            0 & 3
        \end{bmatrix} &&& Q &= \begin{bmatrix}
            1 & 0 \\
            0 & 1
        \end{bmatrix}
    \end{align*}
    We have 
    \[AB = \begin{bmatrix}
        2 & 0 \\
        0 & 6
    \end{bmatrix}\]
    and 
    \[A'B' = \begin{bmatrix}
        4 & 0\\
        0 & 3
    \end{bmatrix}\]
    The eigenvalues of $AB$ do not correspond with $A'B'$, so this operation is not well-defined. 
    \item We have $A \sim B$ means $\det(A) = \det(B)$. So, we want to show that 
    \[\begin{cases}
        A \sim A' \\
        B \sim B' \\
    \end{cases} \implies 
        A + B \sim A' + B'
    \]
    Assume $A \sim A'$ and $B \sim B'$, then $\det(A) = \det(A')$ and $\det(B) = \det(B')$. If $\det(A) = \det(A')$, then $A' = cA$ for some scalar $c$. Similarly for $B$, we have $B' = dB$ for some scalar $B$. 
    
\end{enumerate}
\section{}

\chapter{Facts About Groups, Abelian Groups, Isomorphisms}
\chapter{}
\section{Question 1}
Suppose $\alpha,\beta \in G$, and $\alpha^2 = \epsilon$
$$\alpha\beta = \beta^{-1}\alpha \iff \beta\alpha = \alpha\beta^{-1} \iff (\alpha\beta)^2 = \epsilon \iff (\beta\alpha)^2 = \epsilon$$
\begin{align*}
    \alpha\beta &= \beta^{-1}\alpha\\
    \beta(\alpha\beta)\beta^{-1} &= \beta(\beta^{-1}\alpha)\beta^{-1}\\
    \beta\alpha &= \alpha\beta^{-1}\\
    \alpha(\beta\alpha)\beta &= \alpha(\alpha\beta^{-1}) \beta\\
    \alpha\beta\alpha\beta &= \epsilon\tag{$\alpha\alpha=\alpha^2=\epsilon$}\\
    \alpha\beta\alpha\beta &=\epsilon\ \implies ( \alpha\beta)^2 = \epsilon\\
    \beta(\alpha\beta\alpha\beta)\beta^{-1} &= \beta(\epsilon)\beta^{-1}\\
\end{align*}

\section{Question 2}
Suppose that $G$ is a group and $S \subseteq G$. Show that $\langle S\rangle$ is a subgroup.
\begin{center}
    \textit{Note: $\bangles{S}$ is the set of any product of elements in $S$ and/or their inverses.}
\end{center}
\textbf{Subgroup Test}\\
If $x,y \in \langle S \rangle$, then 
$$x = (product \ of \ some \ elements \ in \ S \ or \ inverses)$$
$$y = (product \ of \ some \ elements \ in \ S \ or \ inverses)$$
So,
$$xy = (products \ of \ elements \ in \ S) \cdot (products \ of \ elements \ in \  S) $$
If $x \in \bangles{S}$, then 
$$x = (product \ of \ inverses \ in \ S \ in \ reverse \ order)$$
Therefore,
$$x^{-1} = (product \ of \ some \ elements \ in \ S \ or \ inverses)$$
This is in $\bangles{S}$.\\
If $S$ is non-empty, then $\bangles{S}$ is nonempty. (Since $S \subseteq \bangles{S}$).\\
If $S = \emptyset$ then the empty product is $\epsilon$, so $\epsilon \in \bangles{S}$
\section{Question 3}
Suppose $G$ is a group, $\phi$ is an automorphism of $G$ if 
\begin{itemize}
    \item $\phi: G \rightarrow G$ is a bijection
    \item $\phi(xy) = \phi(x)\phi(y)$
\end{itemize}
$aut(G)$ is the set of all automorphisms on $G$.
\begin{itemize}
    \item \textbf{\purple{Closed:}}\\
    \begin{center}
        $$\alpha, \beta \in aut(G)$$\\
        $\alpha,\beta$ are isomporhisms $G \rightarrow G$ \\
        $\alpha \circ \beta$ is an isomorphism $G \rightarrow G$
    \end{center}
    \item \textbf{\purple{}}
    \item \textbf{\purple{Associative:}} Composition is always associative.
    \begin{proof}
    Consider $((\alpha \circ \beta)\circ\gamma)(x)$
    \begin{align*}
        ((\alpha \circ \beta)\circ\gamma)(x) & = (\alpha\circ\beta)(\gamma(x))\\
        & = \alpha(\beta(\gamma(x))
    \end{align*}
     \end{proof}
    \item \textbf{\purple{}}
    \item \textbf{\purple{Inverses:}}
    \begin{center}
    $$\phi \in aut(G)$$
    $\phi$ is isomorphism $G \rightarrow G$
    TBC
            
    \end{center}
\end{itemize}

\section{Question 4}
$H_1$ and $H_2$ are subgroups of $G$, with $H_1 \cap H_2 = \{\epsilon\}$. Show $|G| \geq |H_1|\cdot|H_2|$\\[3ex]
\textbf{Solution:}\\[2ex]
\indent
\textbf{Claim:} Instead of taking $(x,y)$, take $xy$ where $x \in H_1$, $y \in H_2$ are all distinct. Let $x, x' \in H_1$, $y,y' \in H_2$. 
\begin{proof}
Suppose $xy = x'y'$, then 
\begin{align*}
    \blue{(x')^{-1}}(xy)\blue{y^{-1}} &= \blue{(x')^{-1}}(x'y')\blue{y^{-1}}\\
    (x')^{-1}x &= y'y^{-1}\\
    (x')^{-1}x &\in H_1\\
    y'y^{-1}&\in H_2\\
    \therefore (x')^{-1}x = &\epsilon = y'y^{-1}
\end{align*}
So $x = x'$ and $y = y'$
\end{proof}

\section{Question 5}
Lattice of subgroups of symmetries of rectangle \\
\textit{Insert graphics}


\chapter{}
\section{Question 1}
\subsection{(a)}
\subsection{(b)} $G$ cyclic $g \in G$ . Is it true that $|g|$ divides $|G|$. \\[3ex]
\textbf{Solution:} $G = \langle a \rangle$ for some a. so $g = a^k$ for some $k$. Then 
$$|g| = |a^k| = \frac{}{}$$
\subsection{(c)}
$G$ is cyclic $d$ divides $|G|$. Is it true that $G$ has a subgroup of order $d$?\\[3ex]
\textbf{Solution:} Suppose $|G| = n$ and $n = d\cdot k$ for some $k$. $G$ is cyclic, so we know 
$$G = \bangles{a}$$
so $|g| = n$, consider $g^k$ order is 
$$\frac{k}{gcd(n,n} = \frac{n}{k} = d $$
So $\bangles{g^k}$ is a subgroup
\subsection{(d)} $H,K$ subgroups of cyclic group $G$ with $|H| = |K|$. Is it true that $H = K$?\\[3ex]
\textbf{Solution:} Consider if the group is finite, so $|G| = n < \infty$. So $G = \bangles{g}$, then $H,K$ are also cyclic. $$H = \bangles{g^r}$$
$$H = \bangles{g^s}$$
For some $s,r \in \ints$. $|H| = |K| = m$ so $|g^r| = |g^s| = m$, then
$$\frac{n}{gcd(n,r)} = \frac{n}{gcd(n,s)} = m$$
Recall,
$$H = \{g^r, g^{2r},g^{3r}, \ldots, g^{mr} \}$$
$$K = \{g^s, g^{2s}, g^{3s}, \ldots, g^{ms}\}$$
We want to show that $s = tr$
$$gcd(n,r) = gcd(n,s) = d =\frac{n}{m}$$
$$\begin{cases}
dr' = r\\
ds' = s
\end{cases} \rightarrow d = \frac{r}{r'} = \frac{s}{s'} = \frac{n}{m}$$
$$r = \frac{r'}{s'} s$$
We know $(g^s)^m = \epsilon$. Consider $\{x: x^m = \epsilon\} = \{g^{dq} : q \in \ints\}$. This set has the size $m$, contains $H$ and $K$. \\[2ex]
\textbf{Summary:} This was a theorem since last class. If $|G| = n$ cyclic and $H$ subgroup of of order $m = \frac{n}{d}$ for some $d$,
\begin{align*}
    H &= \{x: x^m = \epsilon\} \\
    &= \{g^d, g^{2d}, \dots, g^{md}\}
\end{align*}

\subsection{(e)}
$H,K$ cyclic subgroups of $G$. Is it true that $|H \cap K|$ divides $gcd(|H|,|K|)$? \\[3ex]
\section{Question 2}
$G$ is a group and define by $\Gamma(G) = \{g \in G: \bangles{g} = G\}$. Is $\Gamma(G)$ a subgroup of $G$?\\[3ex]
\textbf{Solution:}
No, $\epsilon \not\in \Gamma(G)$ so it is not a subgroup unless $|G| = 1$. Also, is the subgroup $\Gamma(G)$ could be empty. Does $x,y \in \Gamma(G) \implies xy \in \Gamma(G)$? And $x \in \Gamma(G) \implies x^{-1} \in \Gamma(G)$? 
$$\bangles{x} = \{x, x^2,x^3, \dots, x^m\}$$
$$\bangles{x^{-1}} = \{x^{-1}, x^{-2},x^{-3}, \dots, x^{-m}\}$$
So the inverse exists. \\[2ex]
\textbf{Observations:} Let $\{S\}$ denote the union of all "proper" subgroups of $G$. Then
$$\Gamma(G) = G \setminus S$$
Since any element that generates $G$ won't be in any subgroup $S$.
\section{Question 3}
\subsection{(a)}
Find order of $(3,4) \in \ints_7 \times \ints_{12}$
$$\bangles{34} = \{34,68, 90 = 20, 54, 88=18, 40, 04, 38, \ldots \}$$
$$k \cdot 3 \equiv 0 \ (mod \ 7) \iff 7 \mid k$$
$$k \cdot 4 \equiv 0 \ (mod \ 12) \iff 3 \mid k$$
$$k \cdot (3,4) \equiv 0 \ (mod \ (7,12)) \iff 21 \mid k$$
So the order of $(3,4)$ is 21.
\subsection{(b)}
Find order of $(1,1) \in \ints_m \times \ints_n$
$\bangles{11} = \{11,22,33, \ldots\}$
$$k \cdot 1 \equiv 0 \ (mod \ m) \iff m \mid k$$
$$k \cdot 1 \equiv 0 \ (mod \ n) \iff n \mid k$$
$$k \cdot (1,1) \equiv 0 \ (mod \ (m,n)) \iff lcm(n,m) \mid k$$
\subsection{(c)} Find the order of $(a,b) \in \ints_m \times \ints_n$\\[3ex]
\textbf{Solution:}
$$k \cdot a \equiv 0 \ (mod \ m) \iff m \mid ka$$
$$\frac{m}{gcd(m,n)} \mid k\frac{a}{gcd(m,n)}$$
$$k \cdot b \equiv 0 \ (mod \ n) \iff n \mid kb$$
$$\frac{m}{gcd(m,n)} \mid k\frac{b}{gcd(m,n)}$$
Note, $\frac{m}{gcd(m,n)}$ and $k\frac{a}{gcd(m,n)}$ are co-prime, and similarly for $b$.
$$k \cdot (1,1) \equiv (0,0) \ (mod \ (m,n) \iff \frac{n}{gcd(m,n)} \mid k\frac{a}{gcd(m,n)} \wedge \frac{m}{gcd(m,n)} \mid k\frac{b}{gcd(m,n)}$$
$$\iff lcm\left(\frac{n}{gcd(m,n)}, \frac{m}{gcd(m,n)}\right) \mid k$$
\subsection{(d)}
Suppose $\ints_m \times \ints_m$ is cyclic, find all generators.\\[3ex]
\textbf{Solution:}
$$|G| = mn$$
Order of $(a,b) \leq$ order of $(1,1) = lcm(m,n)$. We know it is cyclic $\iff gcd(m,n) = 1$. 
$$|(a,b)| = mn = lcm\left(\frac{n}{gcd(m,n)}, \frac{m}{gcd(m,n)}\right) $$
$$\iff gcd(m,n) = gcd(n,a) = 1$$
\subsection{(e)}
Find a formula for the order of $(x,y)$ in $G \times H$. 
$$(xy)^k = \epsilon \iff (x^k, y^k) = (\epsilon_G, \epsilon_H)$$
$|x|$ in $G$ divides $k$ and $|y|$ in $H$ divides $k$. Therefore $k$ is a multiple of $lcm(|x|, |y|)$
$$\therefore |(x,y)| = lcm(|x|,|y|)$$

$$\implies \impliedby \rightarrow \hookleftarrow \xhookrightarrow{}$$

$$\begin{pmatrix}
1 & 2 & 3\\
1 & 2 & 3
\end{pmatrix}$$

\chapter{Cosets, Isomporhisms, Inner Automorphisms}
\section{Question 1}
Recall that the “(left) Coset Comparison Theorem” says that for $H$ a subgroup of $G$ and $g_1, g_2 \in G$ the following are equivalent:
\begin{enumerate}[label=(\alph*)]
    \item $g_1H = g_2H$
    \item $Hg_1^{-1} = Hg_2^{-1}$
    \item $g_1H \subseteq g_2H$ (or $g_2H \subseteq g_1H$)
    \item $g_1 \in g_2H$ (or $g_2 \in g_1H$)
    \item $g_2^{-1}g_1 \in H$ (or $g_1^{-1}g_2 \in H$)
\end{enumerate}

In each case, the statement remains true if we swap g1 and g2, hence the parenthesized versions.
In class we showed that (a), (b), (e) were equivalent. Show that they are also equivalent to (c) and (d).
\begin{align*}
    g_1H = g_2H &\iff \{g_1h : h \in H\} = \{g_2h : h \in H\}\\
    &\iff \exists \text{ a bijection } \alpha: H \mapsto H\\
    &g_1h = g_2\alpha(h)\\
    &\iff \exists \text{ a bijection } \alpha : H \mapsto H\\
    &(g_1h)^{-1} = (g_2\alpha(h))^{-1}\\
    &h^{-1}g_1^{-1} = \alpha(h)^{-1}g_2^{-1}\\
    &\iff \exists \text{ a bijection } \beta : H \mapsto H\\
    &kg^{-1} = \beta(k)g_2^{-1}\\
\end{align*}
Where $\beta(k) = \alpha(k^{-1})^{-1}$. Therefore,
\[Hg_1^{-1} = Hg_2^{-1}\]
To prove $g_1H = g_2H \iff g_1 \in g_2H$, 
\begin{align*}
    g_1H = g_2H &\iff \{g_1h : h \in H\} = \{g_2h : h \in H\}\\
    &\iff \exists \text{ a bijection } \alpha: H \mapsto H\\
    &g_1h = g_2\alpha(h)\\
    &g_1 = g_2\alpha(h)h^{-1}\\
    &\implies g_1 \in g_2H
\end{align*}
Then we have 
\begin{align*}
    g_1 \in g_2H &\implies g_1=g_2k  \text{ for some } k \in H\\
    &\implies g_1H = \{g_1h: h \in H\}\\
    &= \{g_2kh: h \in H\} = \{g_2h: h \in H\} = g_2H\\
\end{align*}
Proving $g_1 \in g_2H \iff g_1H \subseteq g_2H$. 
\[g_1 \in g_2H \implies g_1h \in g_2H \ \forall h \in H\]
\[g_1H \subseteq g_2H \implies g_1 \epsilon \in g_2H\]
\begin{center}
    \textbf{Recall:}
    $g_1H = Hg_1 \iff g_1H^{-1 } = H$
    
\end{center}

\section{Question 2}
\begin{align*}
    G = D_4 &= \{\mu^i, \rho^i: 0 \leq i \leq 1 \ \ 0 \leq j \leq 3\}\\
    &= \bangles{\mu, \rho: \mu^2 = \rho^2 \epsilon \ \ \rho\mu=\mu\rho^{-1}}
\end{align*}
\subsection{(a)}
$H = \bangles{\mu}$. Fine the left and right cosets of $H$ in $D_4$. \\[2ex]
The size of the coset is the same as 
\[|H| = |\{\epsilon, \mu\}|\] 
The number of cosets are 
\[\frac{|D_4|}{|H|} = \frac{8}{2} = 4\]
So we have 
\begin{itemize}
    \item $\epsilon H = \{\epsilon, \mu\}$
    \item $\rho H = \{\rho, \rho\mu\} = \{\rho, \mu\rho^3\}$
    \item $\rho^2H = \{\rho^2, \rho^2,\mu\} = \{\rho^2, \mu\rho^2\}$
    \item $\rho^3H = \{\rho^3,\rho^3\mu\} = \{\rho^3, \mu\rho\}$
\end{itemize}
Notice, that some of the left cosets are equal to the right cosets, such as $\epsilon H = H\epsilon$ and $H\rho^2 = \rho^2H$.
\subsection{(b)}
$H = \bangles{p}$. Find the left and right cosets of $H$ in $D_4$. Is $gH = Hg$ for every $g \in G$?
We have 
\[\frac{|D_4|}{|H|} = \frac{8}{4} = 2\]
cosets. The size of the cosets is 
\[|H| = |\{\rho,\rho^2, \rho^3, \rho^4 = \epsilon\}| = 4\]
So we have 
\begin{itemize}
    \item $\rho^3 = \epsilon H = \{\epsilon, \rho, \rho^2, \rho^3\} = H\epsilon$
    \item $\mu\rho^2H = \mu H = \{\mu,\mu\rho,\mu\rho^2, \mu\rho^3\} = H\mu$
\end{itemize}
Therefore, all the left cosets are equal to the right cosets.
\subsection{(c)}
Let $H = \bangles{\rho^2}$. Find left and right cosets of $H$ in $D_4$. Is $gH = Hg$ for every $g \in G$?
Similary, we have 
\[\frac{|D_4|}{|H|} = [D_4 : H]= \frac{8}{2} = 4\]
cosets. The size of the cosets is 
\[|H| = |\{\rho^2, \rho^4 = \epsilon\}| = 2\]
So we have 
\begin{itemize}
    \item $\epsilon H = \{\epsilon, \rho^2\}$
    \item $\mu H = \{\mu, \mu\rho^2\}$
    \item $\rho H = \{\rho,\rho^3\} = H \rho$
    \item $\mu\rho H = \{\mu\rho, \mu\rho^3\}$
    \item $H\epsilon = \{\epsilon, \rho^2\}$
    \item $ H \mu = \{\mu, \rho^2\mu\} = \{\mu, \mu\rho^2\}$ since $(\rho^2)^{-1} = \rho^2$. 
    \item $H \rho = \{\rho, \rho^2\}$
    \item $H \mu\rho = \{\mu\rho, \mu\rho^3\}$
\end{itemize}
Notice that every left coset is equal to its right coset. 
\section{Question 3}
$GL_n(\real)$ is the group of $n \times n$ matrices with real entries. $SL_N(\real)$ is the subgroup of $GL_n(\real)$ consisting of matrices with determinant 1.
\subsection{(a)}
Show $SL_n(\real)$ is a subgroup of $GL_n(\real)$.\\[2ex]
We want to prove that 
\[\text{det} = 1 \implies \text{ invertible so } SL_n(\real) \subseteq GL_n(\real)\]
We'll use the subgroup test. 
\begin{itemize}
    \item \textbf{$SL_n(\real)$ is non empty} since $I \in SL_n(\real)$
    \item \textbf{Closure:} If $A,B \in SL_n(\real)$, is $AB \in SL_n(\real)$? Yes, since $\det(AB) = \det(A)\det(B) = 1$. 
    \item \textbf{Inverses: } If $A \in SL_n(\real)$, is $A^{-1} \in SL_n(\real)$? Yes, since 
    $$\det(A^{-1}) = \det(A)^{-1} = \frac{1}{\det(A)} = 1$$.
\end{itemize}
\subsection{(b)}
Describe the cosets of $SL_n(\real)$ in $GL_n(\real)$. Are left and right cosets the same?\\[2ex]
Let $A\in GL_n(\real)$, then 
\begin{align*}
    gH \rightarrow A \cdot SL_n(\real) &= AB : B \in SL_n(\real)\\
    &= \{AB: \det(B) = 1\}
\end{align*}
So, 
\[\det(AB) = \det(A)\det(B) = \det(A)\cdot 1 = \det(A)\]
If $\det(C) = \det(A)$, is $C \in A \cdot SL_n(\real)$? We have 
\[C = A \cdot A^{-1}C\]
Set $B = A^{-1}C$. Then
\[\det(B) = \frac{1}{\det(A)} \det(C) = \frac{1}{\det(A)} \det(A) = 1\]
Therefore, yes.
\section{Question 4}
Let $\psi: G \rightarrow H$ be an isomoprhism. 
\subsection{(a)}
Show that $\alpha \in$ Aut$(G)$ $\implies \psi \circ \alpha$ is an automorphism. \\[2ex]
$\psi$ and $\alpha$ are bijections, so $\psi \circ \alpha$ is a bijection. We want to check the homomorphism property.
\[(\psi \circ \alpha) (xy) = (\psi \circ \alpha)(x)(\psi \circ \alpha)(y)\]
So,
\begin{align*}
    (\psi \circ \alpha)(xy) &= \psi(\alpha(xy))\\
    &= \psi(\alpha(x)\alpha(y))\\
    &= \psi(\alpha(x))\psi(\alpha(y))\\
    &= (\psi \circ \alpha)(x)(\psi \circ \alpha)(y)
\end{align*}

\chapter{Quotient Groups}
\textbf{Recap:} If $H \triangleleft G$, then define the quotient group $G/H$ as 
\begin{itemize}
    \item Elements of $G/H$ are the cosets $gH$ for $g \in G$
    \item The operation in $G/H$ is $aH \cdot bH = (ab)H$, and $ab \in G$.
\end{itemize}
Also, 
\[H \triangleleft G \implies \begin{rcases}
    a_1H = a_2H \\
    b_1H = b_2H
\end{rcases} \implies (a_1b_1)H = (a_2b_2)H\]
So the operation is well-defined.\\[3ex]
\noindent
\textbf{Example:}
Let $G = \ints$, $K = n\ints$ for $n > 0$. $K \triangleleft G$ because $G$ is abelian. So if we take $a + K = \{a + tn : t \in \ints\} = \{t_n + a: t \in \ints\} =K + a$.
\[G/ K = \text{ cosets } = \left\{0 + n\ints, 1 + n\ints, \ldots,(n-1) + n\ints \right\}\]
So there are $n$ cosets, and the operation is addition of cosets, defined the same way as multplication but with addition. 
\[a + n\ints + b + n\ints = (a+b)+ n\ints\]
\textbf{Note:} 
\begin{align*}
    x + n\ints &= \{x + tn: t \in \ints\}\\
    &= \{x + sn + tn : t \in \ints\}\\
    &= x + sn \in n\ints\
\end{align*}
These are equivalent cosets. Notice, 
\[\ints/n\ints = \{j + n\ints : 0 \leq j \leq n-1\}\]
These cosets are exactly the equivalence classes mod $n$. So
\[\ints/n\ints = \ints_n\]
\textbf{Example:}
$G = S_3$, $K = \bangles{\begin{pmatrix}1 & 2\end{pmatrix}} = \{\epsilon, \begin{pmatrix}1 & 2\end{pmatrix}\}$.\\[2ex]
$G = S_3$, $K = \bangles{\begin{pmatrix}1 & 2 & 3\end{pmatrix}} = \{\epsilon, \begin{pmatrix}1 & 2 & 3\end{pmatrix}, \begin{pmatrix}1 & 3 & 2\end{pmatrix}\} = A_3$. Here $K \triangleleft G$. 
\[G/K = \left\{ \{\epsilon, (123), (132)\}, \{(23),(13),(12)\} \right\} = \{K, (23)K\} = \{(132)K, K(13)\} \text{ etc}\]
\[
\begin{tabular}{c|cc}
    $S_3/A_3$ & $A_3$ & $(12)A_3$\\
    \hline
   $ A_3$ &$ A_3 $& $(12)A_3$\\
    $(23)A_3$ & $(12)$ &$ A_3$
\end{tabular}    
\]
$A_3 \cdot A_3 = A_3$ since $\epsilon A_3 = A_3$, so
\[A_3 \cdot A_3 = \epsilon A_3 \cdot \epsilon A_3 = \epsilon\epsilon A_3 = \epsilon A_3 = A_3\]
From the Cayley Table, we can see that $S_3/A_3 \cong \ints_2$. What does multiplication in $S_3/A_3$ look like? 
\begin{itemize}
    \item $S_3/A_3 \cong \ints_2$
    \item In a crude sense, multiplication in $S_3$ is like addition mod 2 because of the parity of the permutation. So the product of two even permutations has an even parity, and an even and odd results in an odd parity.
\end{itemize}
\textbf{Some Context:} From assingment 4. Say $[G:H] = 2$. Show 
\[x,y \in G \setminus H \implies xy \in H\] 
\[x \in G \setminus H \implies x^{-1} \in H\]
If we consider the quotient group $G/H$, consider the Cayley Table
\[
\begin{tabular}{c|cc}
    $G/H$ & $H$ & $G \setminus H$\\
    \hline
    $H$ & $H$ & $G \setminus H$\\
    $G \setminus H$ & $G \setminus H$ &$H$ \\
\end{tabular}
\] 
So, $x, y \in G \setminus H$, $xHyH = xyH$, therefore $xyH = H$ so $xy \in H$. $x \in G \setminus H$. So $x H x^{1}H = x x^{-1}H = \epsilon H = H$. Since $[G : H] = 2$, we have $H \triangleleft G$.
\section{Question 1}
Let $K \leq G$. 
\subsection{Part (a)}
Show $K$ is a normal subgroup if and only if $\forall x \in G, y\in K$, $\exists y' \in K$ such that $xy =y'x$. \\[2ex]
\textbf{Solution:}
For all $x \in G$, we have the cosets
\begin{align*}
    K \text{ is normal in $G$} &\iff gK = Kg \ \ \forall g \in G\\
    &\iff \{xy :  y \in K\} = \{yx = y \in K\}\\
    &\iff \text{ For some $x \in G$, $y\in K$, then $xy \in Kx$. }
\end{align*}
So $xy = y'x$ for some $y' \in K$ as required.
\subsection{Part (b)}
Using part (a), show $\bangles{\rho}$ is a normal subgroup in $D_n$. Identify $y'$ in terms of $xy$. \\[2ex]
\textbf{Solution} Set
\[K = \bangles{\rho} = \{\epsilon, \rho, \rho^2, \rho^3, \ldots, \rho^{n-1}\}\]
Recall,
\[D_n = \{\mu^i\rho^j: 0 \leq i \leq 1, \ 0 \leq j \leq n - 1\} = \{\rho^j: \ 0 \leq j \leq n - 1\} \cup \{\mu\rho^j: 0 \leq j \leq n-1\}\]
If $x$ is a rotation, so $x = \rho^j$, and $y = \rho^r \in K$, then 
\[xy = \rho^j\rho^r = \rho^{j+r} = \rho^r\rho^j = y'x\]
So we can take $y' = \rho^r$. So for $x = \rho^j$, $y' = y$. If $x = \mu\rho^j$, and $y = \rho^r \in K$, then 
\[xy = \mu\rho^j\rho^r = \mu\rho^r\rho^j = \rho^{-r}\mu\rho^j = y'x\]
So for $x = \mu\rho^j$, $y' = \rho^{-r} = y^{-1}$. Furthermore, $K$ is a normal subgroup in $D_n$ since the elements \emph{psuedo} commute, in otherwords there is $y' \in K$ such that $xy = y'x$.

\section{Question 2}
Show that $[G:H] = 2 \implies$ $H$ is a normal subgroup in $G$.(Prof said this is a final exam question)\\[2ex]
\textbf{Solution:}
We have two cosets, one of which must be $H$ since $\epsilon H = H$ and $\epsilon \in H$. Since the cosets partition $G$, we must have that the other coset is $gH$ for some element $g$, so it must be $g \not\in H$. The same idea applies to the right cosets, we have $H$, $Hg$, so $g \not\in H$. So the two subgroups are percisely $H$, $G \setminus H$. So $gH = Hg$. Therefore the subgroup is normal in $G$. 

\section{Question 3} 
Show that if $H$ is a subgroup of $Z(G)$ then $H$ is a normal subgroup of $H$. \\[2ex]
\textbf{Solution:}
We want to prove that $gH = Hg$ $\forall g \in G$. We have
\[gH = \{gh : h \in H\}\]
But, $\forall h \in H$, these elements commute with every element in $G$ since they are in the center of $G$. 
\[gH = \{gh : h \in H\} = \{hg: h \in H\} = Hg\]
as required.

\section{Question 4}
Find $G$ and normal subgroup $K$ with $K$ not a subgroup of $Z(G)$ and $[G :K] \neq 2$. \\[2ex]
\textbf{Solution:}
Consider the question from the previous quiz with $G = D_8$ and 
\[K = \bangles{\rho^2} = \{\epsilon, \rho^2,\rho^4, \rho^6\}\]
so
\[\rho K = \{\rho, \rho^3, \rho^5, \rho^7\} = K \rho\]
\[\mu K = \{\mu, \mu\rho^2, \mu\rho^4, \mu\rho^6\}\]
\[K\mu = \{\mu, \rho^2\mu, \rho^4\mu,\rho^6 \mu\}\]
Using the fact that $\rho \mu = \mu\rho^{-1}$, we have $\rho^2 \mu = \mu(\rho^2)^{-1} = \mu\rho^6 \in \mu K$. Therefore, 
\[\mu K = K \mu\]
\begin{center}
\textbf{Note:} $Z(G) < K$ does not imply $K$ is normal. Consider $H = \bangles{\mu,\rho^4}$  
\end{center}
\textbf{Question:} Consider $D_n$ and $H = \bangles{\rho^k}$. When is $H \triangleleft D_n$? 

\section{Question 5}
Let $K < H < G$ with $H \neq K$ and $H \neq G$. Suppose $K \triangleleft G$. 
\begin{enumerate}[label=(\alph*)]
    \item Is $H \triangleleft G$?
    \item Is $K \triangleleft H$?
\end{enumerate}
\noindent
\textbf{Solution}:
\begin{enumerate}[label=(\alph*)]
   \item No, counter example. Take $H = S_3$, $G = S_4$, and $K = \{\epsilon\}$. $K$ is always normal since the identity always commutes. We want to check if $H$ is not normal in $G$.
   \item We want $hK = Kh $ $ \forall h \in H$. Yes, because it works for all $g \in G$, and $H < G$, so $h \in G$. 
\end{enumerate}

\section{Question 6} Recall the quaternions,
\[Q_8 = \{\pm 1, \pm i, \pm j, \pm k\}\]
\end{document}
