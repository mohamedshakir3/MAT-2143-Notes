\documentclass{article}
\usepackage[landscape]{geometry}
\usepackage{url}
\usepackage{multicol}
\usepackage{amsmath}
\usepackage{esint}
\usepackage{amsfonts}
\usepackage{tikz}
\usetikzlibrary{decorations.pathmorphing}
\usepackage{amsmath,amssymb}
\usepackage{listings}
\usepackage{colortbl}
\usepackage{xcolor}
\usepackage{mathtools}
\usepackage{amsmath,amssymb}
\usepackage{enumitem}
\usepackage{environ}
\usepackage{pgfplots}
\usepackage{graphicx}
\makeatletter

\newcommand*\bigcdot{\mathpalette\bigcdot@{.5}}
\newcommand*\bigcdot@[2]{\mathbin{\vcenter{\hbox{\scalebox{#2}{$\m@th#1\bullet$}}}}}
\makeatother

\title{CSI 2132 Midterm Cheat Sheet}
\usepackage[brazilian]{babel}
\usepackage[utf8]{inputenc}

\advance\topmargin-.8in
\advance\textheight3in
\advance\textwidth3in
\advance\oddsidemargin-1.5in
\advance\evensidemargin-1.5in
\parindent0pt
\parskip2pt
\newcommand{\hr}{\centerline{\rule{3.5in}{1pt}}}
%\colorbox[HTML]{e4e4e4}{\makebox[\textwidth-2\fboxsep][l]{texto}


\definecolor{blue}{HTML}{A7BED3}
\definecolor{brown}{HTML}{DAB894}
\definecolor{pink}{HTML}{FFCAAF}


\newtheorem{theorem}{Theorem}[section]
\newtheorem{definition}{Definition}[section]
\newtheorem{fact}{Fact}[section]
\newtheorem{prop}{Proposition}[section]
\newtheorem{corollary}{Corollary}[section]


\DeclareRobustCommand{\rchi}{{\mathpalette\irchi\relax}}
\newcommand{\irchi}[2]{\raisebox{\depth}{$#1\chi$}} % inner command, used by \rchi


\tikzset{header/.style={path picture={
\fill[green, even odd rule, rounded corners]
(path picture bounding box.south west) rectangle (path picture bounding box.north east) 
([shift={( 2pt, 4pt)}] path picture bounding box.south west) -- 
([shift={( 2pt,-2pt)}] path picture bounding box.north west) -- 
([shift={(-2pt,-4pt)}] path picture bounding box.north east) -- 
([shift={(-6pt, 6pt)}] path picture bounding box.south east) -- cycle;
},
label={[anchor=west, fill=green]north west:\textbf{#1:}},
}} 

\tikzstyle{mybox} = [draw=black, fill=white, very thick,
    rectangle, rounded corners, inner sep=10pt, inner ysep=10pt]
\tikzstyle{fancytitle} =[fill=black, text=white, rounded corners, font=\bfseries]


\tikzstyle{bluebox} = [draw=blue, fill=white, very thick,
    rectangle, rounded corners, inner sep=10pt, inner ysep=10pt]
\tikzstyle{bluetitle} =[fill=blue, inner sep=4pt, text=white, font=\small]


\tikzstyle{brownbox} = [draw=brown, fill=white, very thick,
    rectangle, rounded corners, inner sep=10pt, inner ysep=10pt]
\tikzstyle{browntitle} =[fill=brown, inner sep=4pt, text=white, font=\small]

\tikzstyle{pinkbox} = [draw=pink, fill=white, very thick,
    rectangle, rounded corners, inner sep=10pt, inner ysep=10pt]
\tikzstyle{pinktitle} =[fill=pink, inner sep=4pt, text=white, font=\small]

\tikzstyle{redbox} = [draw=red!35, fill=white, very thick,
    rectangle, rounded corners, inner sep=10pt, inner ysep=10pt]
\tikzstyle{redtitle} =[fill=red!35, inner sep=4pt, text=white, font=\small]



\NewEnviron{redbox}[1]{
    \begin{tikzpicture}
    \node[redbox](box){%
    \begin{minipage}{0.9\textwidth}
    \BODY
    \end{minipage}};
    \node[redtitle, right=10pt] at (box.north west) {#1};
    \end{tikzpicture}
}

\NewEnviron{brownbox}[1]{
    \begin{tikzpicture}
    \node[brownbox](box){%
    \begin{minipage}{0.9\textwidth}
    \BODY
    \end{minipage}};
    \node[browntitle, right=10pt] at (box.north west) {#1};
    \end{tikzpicture}
}

     

\NewEnviron{bluebox}[1]{%
\begin{tikzpicture}
    \node[bluebox](box){%
        \begin{minipage}{0.9\textwidth}
            \BODY
        \end{minipage}
    };
    
\node[bluetitle, right=10pt] at (box.north west) {#1};
\end{tikzpicture}
}

\NewEnviron{pinkbox}[1]{%
\begin{tikzpicture}
    \node[pinkbox](box){%
        \begin{minipage}{0.9\textwidth}
            \BODY
        \end{minipage}
    };
    
\node[pinktitle, right=10pt] at (box.north west) {#1};
\end{tikzpicture}
}



\NewEnviron{blackbox}[1]{%
\begin{tikzpicture}
    \node[mybox](box){%
        \begin{minipage}{0.3\textwidth}
        \small{
            \BODY
        }
        \end{minipage}
    };
    
\node[fancytitle, right=10pt] at (box.north west) {#1};
\end{tikzpicture}
}


\begin{document}

\begin{center}
\large{\textbf{MAT 2143 Midterm Summary Sheet}}\\
\end{center}
\begin{multicols*}{3}
\begin{blackbox}{Equivalence Classes and Relations}
    \begin{bluebox}{Equivalence Relations}
        \textit{$\sim$ is an equivalence relation on a set $X$ if it is
        \begin{itemize}
            \item \textbf{Reflexive:} $x \sim x$ $\forall x \in X$
            \item \textbf{Symmetric:} $x \sim y \iff y \sim x$ $\forall x,y \in X$
            \item \textbf{Transitive:} $x \sim y \wedge y \sim z \implies x \sim z$ $\forall x,y,z \in X$
        \end{itemize}
        }
    We say $\approx$ is a refinement of $\sim$ if $a \approx b \implies a \sim b$ $\forall a,b \in X$ 
    \end{bluebox}
    An \emph{equivalence class} is denoted by\\[-2ex]
    \[[x] = \{y \in X: x \sim y\}\]\\[-4ex]
    \begin{redbox}{Theorems}
    \raggedright
    \textbf{Theorem.}\textit{ Let $X$ be a set with an equivalence relation. Then }\\[-1ex]
    $$[x] \cap [y] \neq \emptyset \implies [x] = [y]$$
    \textbf{Theorem.} Let $X$ be a set with an equivalence relation. Then the equivalence classes form a partition of $X$.\\[1ex]
    \textbf{Theorem.} Let $R_j$ form a partion of $X$. Say that $x \sim y$ means $x,y \in R_j$ for some $j$. Then $\sim$ is an equivalence relation on $X$.\\[1ex]
    \end{redbox}
    \begin{brownbox}{Operations}
    An operation is \emph{well-defined on equivalence classes} if 
    \[\begin{rcases}
    x \sim y\\
    w \sim z
    \end{rcases} \implies x \cdot w \sim y \cdot z\]
    Or equivalently, 
    \[\begin{rcases}
        [x] = [y]\\
        [w] = [z]
    \end{rcases} \implies [x \cdot w] = [y \cdot z]\]
    \textbf{Example:} $X = \mathbb{R} \times \mathbb{R}$ $(a,b) \sim (c,d)$ means $a^2 + b^2 = c^2 + d^2$ Is addition well defined? Let\\[-1ex]
    \[\begin{cases}
        (a,b) \sim (c,d)\\
        (e,f) \sim (g,h)
    \end{cases} \implies \begin{cases}
        a^2 + b^2 = c^2 + d^2\\
        e^2 + f^2 = g^2 + h^2
    \end{cases}\]
    Then\\[-3ex]
    \[\begin{cases}
        (a,b) + (e,f) = (a+e, b+f)\\
        (c,d) + (g,h) = (c+g, d+h)
    \end{cases}\]
    Now we have to check if\\[-1.5ex]
    \[(a+e)^2 + (b+f)^2 = (c+g)^2 + (d+h)^2\]
    \end{brownbox}\\[-2ex]
\end{blackbox}
\begin{blackbox}{Number Theory}
\raggedright
\textbf{Fact:}\textit{ Every Non-empty $S \subseteq N$ has a minimum element $d$ in $S$}\\[0.5ex]
\textbf{Prop:} Let $a, b \in \mathbb{Z}$ with $b > 0$, then $\exists ! q,r \in \mathbb{Z}$ with $a = bq + r$ for $0 \leq r < b$\\[0.75ex]
\begin{bluebox}{GCD}
    \textbf{Definition:} \textit{ Let $a,b \in \mathbb{Z}$, if $d$ is a positive integer with}\\[-4ex]
    \begin{itemize}
        \item $d \mid a$ and $d \mid b$
        \item if $c \mid a$ and $c \mid b$, then $c \mid d$
    \end{itemize}
    \textit{then d is the $\gcd$ of $a$ and $b$}\\[0.5ex]
    \textbf{Theorem:} \textit{For every $a,b \in \mathbb{Z}$, $\exists ! d = \gcd(a,b)$. Furthermore, $\exists x,y \in \mathbb{Z}$ such that $d =   ax+ by$. Furthermore $d$ is the largest common divisor of $a,b$}\\[0.5ex]
    \textbf{Corollary:} \textit{$\gcd(a,b) = 1$ $\implies$ $\exists x,y$ s.t $ax + by = 1$}\\[0.5ex]
    \textbf{Corollary:} \textit{$\gcd(a,b) = d$ $\implies$ $\{ax + by: x,y \in \mathbb{Z}\} = d\mathbb{Z}$}
\end{bluebox}
\begin{redbox}{LCM}
    \textbf{Definition:} \textit{let $a,b \in \mathbb{Z}$ if $m$ is a positive integer with 
    \begin{itemize}
        \item $a \mid m$ and $b \mid m$
        \item if $a \mid n$ and $b \mid n$, then $m \mid n$
    \end{itemize}
    then $m$ is a \emph{lcm} of $a,b$.}\\[1ex]
    \textbf{Theorem:} \textit{ For every $a,b$ $\exists !$ \emph{lcm} $m$}
\end{redbox}\\[-2ex]
\end{blackbox}
\begin{blackbox}{Cayley Tables}
\[\begin{tabular}{c|cccc}
    $\cdot$ &  $\epsilon$ & $a_1$ &$ a_2 $& $\cdots$ \\
    \hline
    $\epsilon$ & $\epsilon$ & $a_1$ & $a_2$ & $\cdots$ \\
    $a_1$ & $a_1$ & $\cdots$ & $\cdots$ & $\cdots$\\
    $a_2$ & $a_2$ & $\cdots$ & $\cdots$ & $\cdots$\\
    $\vdots$ & $\vdots$ & $\vdots$ & $\vdots$ & $\vdots$\\
\end{tabular}\]
\textbf{Properties:}
\begin{itemize}
    \item Symmetric $\implies$ Operation is commutative
    \item row and column is the header $\implies$ corresponding element is the identity
    \item every row has the identity $\implies$ each element an inverse
    \item Only one row and column can match the header (in other words there is only one identity)
    \item Each row and column contains each element \emph{exactly} once (since the group is closed)
\end{itemize}
\end{blackbox}
\begin{blackbox}{Isomorphisms}
\begin{brownbox}{Isomoprhism}
    If $\phi: G \rightarrow H$ is a bijection with $\phi(xy) = \phi(x)\phi(y)$ Then $\phi$ is an isomoprhism and $G,H$ are isomorphic. 
\end{brownbox}
\begin{pinkbox}{Automorphism}
If $\phi: G \rightarrow G$ is an isomorphism, then $\phi$ is an automorphism. We denote the set of all automorphisms as aut$(G)$ 
\end{pinkbox}\\[-2ex]
\end{blackbox}
\begin{blackbox}{Cyclic Groups}
    \textbf{Definition:} \textit{G is cyclic $\iff$ $\exists$ a generator $g \in G$ s.t $G = \langle g \rangle = \{g^k: k \in \mathbb{Z}\}$}
    \raggedright
    The \emph{order} of an element $g \in G$ is the smallest positive integer $n$ with $g^n = \epsilon$\\[1.2ex]
    \begin{bluebox}{Facts and Notation}
        \begin{itemize}
        \item $|g| = $ order of an element, $|g| = \infty \iff g^k \neq \epsilon$ $\forall k \in \mathbb{Z}$
        \item $\{k: g^k = \epsilon\} = |g| \cdot \mathbb{Z}$, so $g^k = \epsilon \iff |g| \mid k$
        \item $|x| = |y| \iff (x^k = \epsilon \iff y^k = \epsilon)$ 
        \item $G$ is cyclic $\implies$ $G$ is abelian
        \item $G$ is cyclic $\implies$ All subgroups of $G$ are cyclic 
        \item $G$ is cyclic with with no subgroups other than $\{\epsilon\} \iff$ $|G| = n$ is prime. \textit{(We say G is cyclic of prime order)}
        \item If $G,H$ are both cyclic, then $G \cong H \iff |G| = |H|$
        \item $|g^k| = \frac{n}{\gcd(n,k)}$
        \item Generators are exactly $\{g^k: \gcd(n,k) = 1\}$
    \end{itemize}   
    \end{bluebox}\\[-2ex]
\end{blackbox}
\begin{blackbox}{Complex Numbers}
    \[ \mathbb{C} = \{a + bi: a,b \in \mathbb{R}\}\]
    \[ \mathbb{C} = \{re^{i\theta}: r,\theta\in \mathbb{R} \text{ s.t } r \geq 0, 0 \leq \theta < 2\pi\}\]
    \[re^{i\theta} = r\cos\theta + ri\sin\theta \implies e^{i\theta} = \cos\theta + i\sin\theta\]
    \[z = re^{i\theta} = re^{i(\theta + 2k\pi)} = -re^{i(\theta + (2k + 1)\pi)}\]
    \[z = re^{i\theta} = r\cos\theta + ir\sin\theta\]
    \[a = r\cos\theta \text{ and } b = ir\sin\theta\]
    $$|z| = |a+bi| = \sqrt{a^2 + b^2} = r$$
    \[\frac{b}{a} = \tan\theta\]
\end{blackbox}
\begin{blackbox}{Roots of Unity and The Circle Group $\mathbb{T}$}
The \emph{nth} root of unity is the solution to $z^n = 1$\\[-1ex]
\[R_n = \{e^{i2\pi\cdot\frac{1}{n}}, e^{i2\pi\cdot\frac{2}{n}}, \ldots, e^{i2\pi\cdot\frac{n}{n}}\} = \langle e^{\frac{i2\pi}{n}} \rangle\]\\[-2ex]
\begin{bluebox}{Circle Group $\mathbb{T}$}
    \[\mathbb{T} = \{z \in \mathbb{C}: |z| = 1\} = \{e^{i\theta}: \theta \in \mathbb{R}\} \leq \mathbb{C}^\times\]\\[-7ex]
    \[R_n \leq \mathbb{T} \leq \mathbb{C}^\times\]
\end{bluebox}
\begin{pinkbox}{R Group}
    \[R = \bigcup_{n=1}^\infty R_n = \{e^{\frac{2\pi ij}{n}}: 0 \leq j <n, \ n \geq 1\}\]
    \textbf{Properties:}
    \begin{itemize}
        \item $|z|$ is finite $\forall z \in R$
        \item $|R|$ is infinite
        \item R is abelian but \emph{not} cyclic
        \item Every finite subset is contained in a finite subgroup
        \item Every finite subgroup is cyclic
        \item Every infinite subgroup is not cyclic
        \[R = \langle \{e^{\frac{2\pi i}{n}}: n \geq 1\} \rangle = \langle \{e^{\frac{2\pi i}{n}}: n \geq k\} \rangle\]
        For any $k$
    \end{itemize}
\end{pinkbox}\\[-4ex]
\begin{center}
    \textbf{Subgroup Hierarchy:}
\[R_n < R < \mathbb{T} < \mathbb{C^\times}\]
\end{center}
\end{blackbox}
\begin{blackbox}{Symmetric Group}
$\Omega$ is some set, a \emph{permutation} of $\Omega$ is a bijection $\Omega \mapsto \Omega$. $S_{\Omega} = $ the set of all permutations of $\Omega$, which is called the \emph{symmetric group} $S_n$. $S_n = S_{\Omega}$ for $\Omega = \{1,2,\ldots, n\}$. so $|\Omega| = n$.\\
A subgroup of $S_n$ is called a permutation group.  
\begin{redbox}{Theorems}
    \begin{itemize}
        \item $S_\Omega$ with the operation of compositions is a group
        \item $|S_n| = n!$
    \end{itemize}
\end{redbox}
\begin{brownbox}{Cycle Notation}
    If $\sigma \in S_n$ then
    \[\sigma = \begin{pmatrix}
        1 & 2 & \cdots& n\\
        \sigma(1) & \sigma(2)&  \cdots & \sigma(n)
    \end{pmatrix}\]
\end{brownbox}\\[-2ex]   
\end{blackbox}
\begin{blackbox}{Cycles}
$\sigma \in S_n$ is a cycle if $\exists a_1, \ldots, a_k$ such that 
\[\begin{cases}
    \sigma(a_j) = a_{j+1}\\
    \sigma(a_k) = a_1\\
    \sigma(x) = x, \ x \neq a_j
\end{cases}\]\\[-4ex]
    \begin{bluebox}{Cycle Order}
        \begin{itemize}
            \item A $k$-cycle has $a_1, \ldots, a_k$ terms
            \item 2-cycles are called \emph{transpositions}
        \end{itemize}
    \end{bluebox}
    \begin{brownbox}{More Notation}
        \textbf{Two-Line Notation:}\\[-1ex]
        \[\sigma = \begin{pmatrix}
            1 & 2 & 3 & 4 & 5\\
            3 & 2 & 5 & 1 & 4
        \end{pmatrix}\]\\[-3ex]
        \textbf{One-Line Notation:}\\[-1ex]
        \[\sigma = \begin{pmatrix}
            1 & 3 & 5 & 4
        \end{pmatrix}(2)(6) = \begin{pmatrix}
            1 & 3 & 5 & 4
        \end{pmatrix}\]
        \[\sigma^{-1} = \begin{pmatrix}
            4 & 5 & 3 & 1
        \end{pmatrix}\]
    \end{brownbox}
    \begin{pinkbox}{Supports}
        The support of a permutation $\pi$ is $\{x: \pi(x) \neq x\}$. Permutations are \emph{disjoint} if their supports are disjoint. 
        \textbf{Example:}\\[-2ex]
        \[\sigma = \begin{pmatrix}
            1 & 2 & 3 & 4 & 5 \\
            5 & 2  & 3 & 1 & 4
        \end{pmatrix}, \text{ support}(\sigma) = \{1,4,5\}\]
    \end{pinkbox}
    \begin{redbox}{Cycle Types \& Order}
        The \emph{cycle type} of a permutation $\pi$ is the list of the lengths of its disjoint cycles. The order is the lcm of the cycle types.  \\
        \textbf{Example:} List all the possible orders and cycle-types of permutations in $S_7$\\[-2ex]
        \[\begin{tabular}{c|c}
              Cycle-Type & Order \\
              \hline
              7 & 7\\
              6 & 6\\
              5,2 & 10\\
              5 & 5\\
              4,3 & 12\\
              4,2 & 4\\
              4 & 4\\
              3,3 & 3\\
              3,2,2 & 6\\
              3,2 & 6\\
              3 & 3\\
              2,2,2 & 2\\
              2,2 & 2\\
              2 & 2 \\
              1 & 1\\
        \end{tabular}\]
    \end{redbox}\\[-2ex]
\end{blackbox}
\begin{blackbox}{Dihedral Group}
$D_n$ is the group of symmetries of a regular $n$-gon with 
\begin{itemize}
    \item $\rho =$ reflection by $\frac{1}{n}$ circle $ = \begin{pmatrix}
        1 & 2 & \cdots & n
    \end{pmatrix}$
    \item $\mu=$ reflection through corner 1 $=$\\[-2ex]
\end{itemize}
$\begin{cases}
        (1)\begin{pmatrix}2 & 2m\end{pmatrix}\begin{pmatrix}3 & 2m-1\end{pmatrix}\ldots
        \begin{pmatrix}m & m+2\end{pmatrix}(m+1)\\
        (1)\begin{pmatrix}2 & 2m+1\end{pmatrix}\begin{pmatrix}3 & 2m\end{pmatrix}\ldots
        \begin{pmatrix}m+1 & m+2\end{pmatrix}
    \end{cases}$\\[1ex]
    For $n = 2m$, and $m = 2m+1$ respectively. \\[1ex]
    \[D_n = \{\mu^i\rho^j\} = \{\rho^j\mu^i\}\]
    \textbf{Theorem:} $D_n$ is a subgroup of $S_n$
\end{blackbox}
\begin{blackbox}{Conjugation}
    $\sigma, \pi \in S_n$, we say $\pi$ is conjugated by $\sigma$ for $\sigma\pi\sigma^{-1}$. Suppose $\pi(i) = j$, then\\[-2ex]
    \[\pi(i) = j \iff (\sigma\pi\sigma^{-1})(\sigma(i)) = \sigma(j)\]
    \textbf{Proposition:} $\alpha, \beta \in S_n$ have the same cycle type $\iff$ $\beta = \sigma \alpha \sigma^{-1}$ for some $\sigma \in S_n$.
\end{blackbox}
\begin{blackbox}{Important Facts/Theorems}
    \textit{*Note: Some of these are repeats but are very important}
    \raggedright
    \begin{itemize}
        \item $|g^k| = \frac{n}{\gcd(n,k)}$
        \item If $G,H$ are both cyclic, then $G \cong H \iff |G| = |H|$
        \item Cyclic $\implies$ Abelian
        \item Disjoint permutations commute
        \item $x \in $ support$(\pi) \implies \pi(x), \pi(\pi(x)), \ldots \in$ supp($\pi$)
        \item Order of a permutation is the lcm of the cycle types
        \item Every permutation can be written as products of disjoint cycles
        \item $S_n$ is generated by the set of all cycles
        \item $k$-cycles can be written as the product of $k - 1$ transpositions
        \item The set of all transpositions generates $S_n$, so $S_n = \langle \{\begin{pmatrix}
            a & b
        \end{pmatrix}: 1 \leq a < b \leq n\}$
        \item The following are minimal generating sets of $S_n$
        \[\{\begin{pmatrix}
            1 & a
        \end{pmatrix}: 2 \leq a \leq n\}\]
        \[\{\begin{pmatrix}
            a & a+1
        \end{pmatrix}: 1 \leq a \leq n-1\}\]
        \[\{\begin{pmatrix}
            1 & 2
        \end{pmatrix}, \begin{pmatrix}
            1 & 2 & \ldots & n
        \end{pmatrix}\}\]
        \item If $G$ is abelian and $H$ is not, then they are never isomorphic.
    \end{itemize}
\end{blackbox}
\begin{blackbox}{Lagrange Theorem}
    Let $G$ be a finite group and $H$ be a subgroup of $G$. Then\\[-2ex]
    \[|G| = [G : H] \cdot |H| \implies [G:H] = \frac{|G|}{|H|}\]
    \raggedright
    $[G:H]$ is the number of left cosets of $G$ in $H$.
    \begin{bluebox}{Corollaries}
        \textbf{Corollary:}\\[-2ex]
        \[H <  G \implies |H| \text{ divides } |G|\]
        \textit{Proof. } \\[-2ex]
        $$|H| \cdot [G : H]  = |G|$$
        \textbf{Corollary:}\\[-2ex]
        \[g \in G \implies |g| \text{ divides } |G|\]
        \textit{Proof. }\\[-2ex]
        \[|g| = |\langle {g} \rangle| \ \ |\langle {g} \rangle| \cdot [G : \langle {g} \rangle]\]
        \textbf{Corollary:}\\[-2ex]
        \[|G| \text{ prime } \implies G = \langle a \rangle \ \forall a \neq \epsilon\]
        If 
        \[K < H < G\]
        then 
        \[|G| = [G : H][H : K] |K|\]
        \[[G:K] = [G:H][H:K]\]
    \end{bluebox}\\[-2ex]
\end{blackbox}
\begin{blackbox}{Cosets}
    $H$ is s subgroup of $G$, $g$ is any fixed element in $G$. Then the left coset of $H$ in $G$ is \\[-2ex]
    \[gH = \{gh: h \in H\}\]
    The right coset of $H$ in $G$ is \\[-2ex]
    \[Hg = \{hg: h \in H\}\]\\[-4ex]
    \begin{redbox}{Properties}
    \begin{itemize}
        \item $G$ abelian $\implies$ $gH = Hg$ $\forall g \in G$ $H \leq G$
        \item $H \leq Z(G) \implies gH = Hg$ $\forall g \in G$
        \item $g \in Z(G) \implies gH = Hg$ $\forall H \leq H$  
    \end{itemize}
    \begin{bluebox}{Equivalent Statements}
        Let $H \leq G$ $g_1, g_2 \in G$
        \begin{itemize}
            \item $g_1H = g_2H$ 
            \item $Hg_1^{-1} = Hg_2^{-1}$
            \item $g_1H \subseteq g_2H$ (or $g_2H \subseteq g_1H$)
            \item $g_1 \in g_2H$ (or $g_2 \in g_1H$)
            \item $g_2^{-1}g_1 \in H$ (or $g_1^{-1}g_2 \in H$)
        \end{itemize}
    \end{bluebox}
    \end{redbox}\\[-2ex]
\end{blackbox}
\end{multicols*}
\end{document}
